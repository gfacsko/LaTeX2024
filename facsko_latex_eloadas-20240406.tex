\documentclass[a4paper,12pt]{article}

\usepackage{amsmath}
\usepackage{esint}

\setlength{\parindent}{0pt} % Nincs behúzás
\setlength{\parskip}{6pt}  % Fél sor kihagyás a fejezetek között


\begin{document}

\section{Példa rendes szövegre}

Példaszöveg. Példaszöveg. Példaszöveg. Példaszöveg. Példaszöveg. Példaszöveg. Példaszöveg. Példaszöveg. Példaszöveg. Példaszöveg. Példaszöveg. Példaszöveg. Példaszöveg. Példaszöveg. Példaszöveg. Példaszöveg. Példaszöveg. Példaszöveg. Példaszöveg. Példaszöveg. Példaszöveg. Példaszöveg. Példaszöveg. Példaszöveg. Példaszöveg. Példaszöveg. Példaszöveg. Példaszöveg. Példaszöveg. Példaszöveg. Példaszöveg. Példaszöveg. Példaszöveg. Példaszöveg. {Példaszöveg.}  \textbf{Példaszöveg.}

\section{Matematikai szövegek}

\subsection{\TeX-szerű képletek}

A szénsav kémiai képlete a $H_{2}CO_{3}$ formula. 

A szénsav kémiai képlete a $$H_{2}CO_{3}$$ formula. 

Példák indexelésre: $H_2CO_3$, $H_{2}CO_{3}$, ${H}_{2}C{O}_{3}$.

Pithagorasz-tétele: $$a^{2}+b^{2}=c^{2}$$

\subsection{\LaTeX-szerű képletek}
%\subsection*{\LaTeX-szerű képletek}

\( F_{i} \)

\[ F_{i} \]

Szövegközi képletek \textit{math} környezettel: 
\begin{math}
%F_{i}
\frac{\sqrt[5]{a^3}}{5}
\end{math}
így írhatóak.

Kiemelt képletek \textit{equation} környezettel: 
\begin{equation}
\label{eq:fi}
F_{i}
\end{equation}
így és
\begin{equation}
\label{eq:tort}
\frac{\sqrt[5]{a^3}}{5}
\end{equation}
így írhatóak.

Feljebb példát adtunk egy indexre (\ref{eq:fi}), továbbá egy törtre (\ref{eq:tort}).

\begin{eqnarray}
c & = & \frac{\sqrt{a^3}}{b} \\
F_L & = & q \cdot (\mathbf{v} \times \mathbf{B}) \\
E &=& m c^2
\end{eqnarray}

\begin{eqnarray*}
c & = & \frac{\sqrt{a^3}}{b} \\
F_L & = & q \cdot (\mathbf{v} \times \mathbf{B}) \\
E &=& m c^2
\end{eqnarray*}

\subsection{Matematikai formulák}

Törtek: $\frac{a}{b}$.

Gyökvonás: $\sqrt{a}$, $\sqrt[3]{b}$.

Trigonometria: $\sin x$, $f\left(x\right)=\frac{\sin x}{x}$.

Szimbólumok: $a+b+c=d$, $a<b$, $a \le b$, $a \ge b$, $a \ne b$, $a \sim b$.

Metszet: $A \cap B$, unió: $A \cup B$.

Példák görök betűkre: $\alpha$, $\beta$, $\gamma$,  $\Gamma$, $\Theta$, $\theta$, $\vartheta$.

Nabla: $\nabla \mathbf{B} = 0$.

\subsubsection{Analízis}

Derivált: $$\frac{df}{dx},$$ ahol $f\left(x\right)$ x-től függő fügvény.

Paricális derivált: $$ \frac{\partial g}{\partial x},$$  másodfokú parciális deriváltak:  $$\frac{\partial^2 g}{\partial^2 x},$$ $$\frac{\partial^2 g}{\partial x \partial y},$$ $$\frac{\partial^2 g}{\partial^2 y},$$ ahol $g\left(x,y\right)$ x-től és y-tól függő függvény.

Határozatlan integrál: $$ \int \frac{1}{x} dx.$$

Határozott integrál: $$\int\limits_{-\infty}^{\infty} \frac{\sin x}{x} dx = 1.$$

Többszörös integrálás: $$\iiint g\left(x,y,z\right) dx\,dy\,dz.$$

Körintegrál: $$\oiint\limits_{S} F\left(x,y\right) dx\,dy.$$

\subsubsection{Határérték, összegek, egyebek}

Összegek: $\sum\limits^n_{i=1} \frac{1}{x_i}$, $$\sum\limits^n_{i=1} \frac{1}{x_i}.$$

Products: $\prod\limits^n_{i=1} \frac{1}{x_i}$, $$\prod\limits^n_{i=1} \frac{1}{x_i}.$$

Határértékek: $$\lim\limits_{n \to \infty} a_n.$$

Unió halmazrendszer: $$\bigcup\limits_{i=1}^{n} H_{i}.$$

Metszet halmazrendszer: $$\bigcap\limits_{i=1}^{n} H_{i}.$$

\end{document}
