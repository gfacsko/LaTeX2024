\documentclass[aspectratio=169]{beamer}

\usepackage{cite}
\usepackage{esint}
%\usepackage{natbib}
\usepackage{hologo}
\usepackage{inlinebib}

% ----------------------------------------------------------------------------------------------------------------------------------------

\title{\hfill \includegraphics{MFE.png}\\Bevezetés a \LaTeX\ használatába}

% \subtitle{5-8. előadás}

\author{Dr.~Facskó Gábor, PhD\\\tiny főiskolai adjunktus, tudományos főmunkatárs\\\textit{facsko.gabor@uni-milton.hu}}

\institute{\Tiny Milton Friedman Egyetem, Informatikai Tanszék, 1039 Budapest, Kelta utca 2.\\Wigner Fizikai Kutatóközpont, Űrfizikai és Űrtechnikai Osztály, 1121 Budapest, Konkoly-Thege Miklós út 29-33.\\ \texttt{https://wigner.hu/$\sim$facsko.gabor}}

\date{2024.~május~4.}

\logo{\includegraphics[height=0.9cm]{kekcsik.png}}

% ----------------------------------------------------------------------------------------------------------------------------------------

\begin{document}

\frame{\titlepage}

% ----------------------------------------------------------------------------------------------------------------------------------------

\begin{frame}[fragile,squeeze]
\frametitle{Irodalomjegyzék létrehozása a szövegben}

Az önsanyargató hajlamokkal rendelkezők használják a \texttt{\textbackslash bibitem} parancsot a \texttt{thebibliography} környezetben:
\begin{verbatim*}
\begin{thebibliography}{\textbf{Irodalomjegyzék}}
\bibitem[Bacsardi(2005)]{bacsi2}Bacsardi, L.\ 2005, Acta Astronautica, 57, 224
\bibitem[Bacsardi(2007)]{bacsi1}Bacsardi, L.\ 2007, Acta Astronautica, 61, 151
\end{thebibliography}
\end{verbatim*}

%Idézem Bacsárdi László összes művét. \cite{bacsi1},  \cite{bacsi2}

%\begin{thebibliography}{9}
%\bibitem[Bacsardi(2005)]{bacsi2}Bacsardi, L.~Acta Astronautica, 57, 224, 2005
%\bibitem[Bacsardi(2007)]{bacsi1}Bacsardi, L.~Acta Astronautica, 61, 151, 2007
%\end{thebibliography}
\vfill
\end{frame}

% ----------------------------------------------------------------------------------------------------------------------------------------

\begin{frame}[fragile,squeeze]
\frametitle{Irodalomjegyzék létrehozása a \hologo{BibTeX} segédprogrammal}

\begin{itemize}
\item Használata:
\begin{verbatim*}
\bibliography{facsko_gabor_researchplan} 
\bibliographystyle{plain}
\end{verbatim*}
\item Irodalomjegyzék formátumok: \texttt{plain}, \texttt{unsrt}, \texttt{alpha}, \texttt{abbvr}. Idézés: \texttt{\textbackslash cite\{\}}
\item A \textit{natbib} csomaggal pl.~\textit{plainnat} (ld.~a csomag paramétereit). Idézés: \texttt{\textbackslash citep\{\}} \texttt{\textbackslash citet\{\}} (paraméterek).
\item A \hologo{BibTeX} file tartalma: \texttt{article}, \texttt{book}, \texttt{report}.
\item SAO/NASA Astrophysics Data System (ADS) a kész \hologo{BibTeX} formátumú bibliográfiai adatok lelőhelye:\\\texttt{http://adsabs.harvard.edu/}
\end{itemize}
\vfill
\end{frame}

% ----------------------------------------------------------------------------------------------------------------------------------------

\begin{frame}[fragile,squeeze,allowframebreaks]
\frametitle{Példák \hologo{BibTeX} formátumokra}

\begin{verbatim*}
@ARTICLE{facsko08:_clust,
   author = {{Facsk{\'o}}, G. and {Kecskem{\'e}ty}, K. and 
{Erd{\H o}s}, G. and {T{\'a}trallyay}, M. and 
{Daly}, P.~W. and {Dandouras}, I.},
    title = "{A statistical study of hot flow 
anomalies using Cluster data}",
  journal = {Advances in Space Research},
     year = 2008,
   volume = 41,
    pages = {1286-1291},
      doi = {10.1016/j.asr.2008.02.005}
}
\end{verbatim*}
\vfill

% ----------------------------------------------------------------------------------------------------------------------------------

\begin{verbatim*}
@Book{laakso10:_clust_activ_archiv,
  editor = 	 {{Laakso}, H. and {Taylor}, M. 
and {Escoubet}, C.~P.}
  title = 	 {The Cluster Active Archive},
  publisher = 	 {Springer Science+Business Media B.V.},
  year = 	 2010,
  note = {Astrophysics and Space Science Proceedings, 
ISBN 978-90-481-3498-4}},
  doi = {10.1007/978-90-481-3499-1}
}
\end{verbatim*}
\vfill

% ------------------------------------------------------------------------------------------------------------------------------------

\begin{verbatim*}
@InProceedings{facsko10:_clust_hot_flow_anomal_obser,
   author = {{Facsk{\'o}}, G. and {T{\'a}trallyay}, M. and 
{Erd{\H o}s}, G. and {Dandouras}, I.},
    title = "{Cluster Hot Flow Anomaly Observations 
During Solar Cycle Minimum}",
booktitle = {The Cluster Active Archive, Studying 
the Earth's Space Plasma Environment},
     year = 2010,
   editor = "{Laakso, H., Taylor, M., \& Escoubet, C.~P.}",
    pages = {369-375},
      doi = {10.1007/978-90-481-3499-1\_25}
}
\end{verbatim*}
\vfill
\end{frame}

% ----------------------------------------------------------------------------------------------------------------------------------------

\begin{frame}[fragile,squeeze,allowframebreaks]
  \frametitle{Komplex dokumentumok - Cikk írása, az article formátum}
  
\begin{itemize}
\item Az \textit{article} osztály
\begin{verbatim*}
\documentclass[a4paper,12pt,draft]{article}
\usepackage[utf8]{inputenc}
\usepackage[magyar]{babel} % Másolás!!!
\begin{document}
\title{Kilencedik}
\author{Facskó Gábor}
\date{\today}
\maketitle
Próba. Próba.
\end{document}
\end{verbatim*}
\item Próbáljuk ki a \textit{10pt, 11pt, twocolumn, draft} paramétereket.
\end{itemize}



% ----------------------------------------------------------------------------------------------------------------------------------------

\pagebreak
\begin{itemize}
\item A teljes címlap kézzel is megtervezhető.
\begin{verbatim*}
\begin{titlepage}
\end{verbatim*}
\item Fontos része a cikknek az absztrakt/kivonat. De lehet ''Executive Summary'' is.
\begin{verbatim*}
\begin{abstract}
\renewcommand{\abstractname}{Executive Summary}
\end{verbatim*}
\item Bekezdések
\begin{verbatim*}
\setlength{\parindent}{1cm}
\setlength{\parskip}{6pt}
\end{verbatim*}
\item Lábjegyzet. 
\begin{verbatim*}
\footnote{Lábjegyzet.}
\end{verbatim*}
\item Fejléc, lábléc: \textit{plain, empty, headings, myheadings}
\begin{verbatim*}
\pagestyle{plain}
\thispagestyle{headings}
\end{verbatim*}
\end{itemize}
\vfill
\end{frame}

% ----------------------------------------------------------------------------------------------------------------------------------------

\begin{frame}[fragile,squeeze,allowframebreaks]
\frametitle{Komplex dokumentumok - Beszámolók, a report formátum}

\begin{itemize}
\item Címlap, oldalformátumok: ugyanaz, mint az article, de a címlap külön oldalon van.
\item Részek, Fejezetek, bekezdések
\begin{verbatim*}
\chapter{}
\section{}
\subsection{}
\subsubsection{}
\paragraph{}
\subparagraph{}
\end{verbatim*}
Csillaggal: számozás nélkül
\item Tartalomjegyzék, ábra- és táblázatjegyzék.
\begin{verbatim*}
\tableofcontents
\listoffigure
\listoftable
\end{verbatim*}
\end{itemize}
\vfill

\pagebreak  % -----------------------------------------------------------------------------------------------------------------------

\begin{itemize}
\item Tartalomjegyzék, ábra- és táblázatjegyzék - cont'd
\begin{verbatim*}
\addcontentsline{toc}{chapter}{Ábrajegyzék}
\end{verbatim*}
\item Tárgymutató.
\begin{verbatim*}
\usepackage{makeidx}
\makeindex
\index{Próba}
\printindex
makeindex kilencedik.idx
\end{verbatim*}
Használjuk az \textit{\textbackslash addcontentsline\{\}}\ parancsot.
\end{itemize}
\vfill
\end{frame}

% ----------------------------------------------------------------------------------------------------------------------------------------

\begin{frame}[fragile,squeeze]
\frametitle{Komplex dokumentumok - Szakdolgozatok, könyvek, a book formátum}

\begin{itemize}
\item A book formátum ugyanazt tudja, mint a report, de nem csak egy oldalformátummal dolgozik.
\item Rakjunk össze egy könyvet
\begin{verbatim*}
\renewcommand{\thepage}{}
\pagenumbering{roman}
\pagenumbering{arabic}
\end{verbatim*}
\end{itemize}
\vfill
\end{frame}

% ----------------------------------------------------------------------------------------------------------------------------------------

\begin{frame}[fragile,squeeze]
\frametitle{Komplex dokumentumok - Poszter és a prezentáció készítés \LaTeX-hel}

\begin{itemize}
\item Ne készítsünk posztert és prezentációkat \LaTeX-hel. Jobb a Power Point.
\item A \textit{slides} dokumentum osztály. Nincs benne irodalomjegyzék.
\item A \textit{beamer} dokumentum osztály. Nem ismerem.
\end{itemize}
\vfill
\end{frame}

% ----------------------------------------------------------------------------------------------------------------------------------------

\begin{frame}[fragile,squeeze]
\frametitle{Grafika a \LaTeX-ben - egyenesek, vektorok}

\begin{itemize}
\item \textit{picture} környezet: egyszerű ábrák a \LaTeX-ben.
\item Sok program képes \LaTeX\ formátumban menteni.
\begin{verbatim*}
\setlength{\unitlength}{1mm} % Unit beállítás
\begin{picture}(30,35) % képméret
\thicklines % vonal választás
%\thinlines
\put(-5,0){\vector(1,0){30}} % Vektor
\put(0,-5){\vector(0,1){35}}
\put(0,0){\line(4,5){24}} % Vonal
\end{picture}
\end{verbatim*}
\end{itemize}
\vskip -3 true cm
\hskip 5 true cm
Kirajzol egy koordinátarendszert: \ \ 
\setlength{\unitlength}{1mm}
\begin{picture}(30,35)
\thicklines
%\thinlines
\put(-5,0){\vector(1,0){30}}
\put(0,-5){\vector(0,1){35}}
\put(0,0){\line(4,5){24}}
\end{picture}
\vfill
\end{frame}

% ----------------------------------------------------------------------------------------------------------------------------------------

\begin{frame}[fragile,squeeze]
\frametitle{Grafika a \LaTeX-ben - körök, oválisok}

\begin{itemize}
\item Körök, kitöltés csillaggal.
\begin{verbatim*}
\setlength{\unitlength}{1mm}
\begin{picture}(30,35)
\thicklines %\thinlines
\put(100,25){\vector(1,0){10}}
\put(100,25){\circle{12}}
\put(100,40){\circle*{5}}
\put(90,10){\oval(20,10)[tl]}
\put(100,5){\oval(20,10)}
\put(110,0){\oval(20,10)[br]}
\end{picture}
\end{verbatim*}
\setlength{\unitlength}{1mm}
\vskip -4 true cm 
\begin{picture}(30,35)
\thicklines %\thinlines
\put(100,25){\vector(1,0){10}}
\put(100,25){\circle{12}}
\put(100,40){\circle*{5}}
\put(90,10){\oval(20,10)[tl]}
\put(100,5){\oval(20,10)}
\put(110,0){\oval(20,10)[br]}
\end{picture}
\end{itemize}
\vfill
\end{frame}

% ----------------------------------------------------------------------------------------------------------------------------------------

\begin{frame}[fragile,squeeze,allowframebreaks]
\frametitle{Grafika a \LaTeX-ben - Dobozok, szövegek az ábrában.}

\begin{itemize}
\item Dobozok, szöveggel: 
\begin{verbatim*}
\makebox(x,y)[poz]{szöveg}
\framebox(x,y)[poz]{szöveg}
\end{verbatim*}
poz: l,r,t,b: balra, jobbra, fel, le és kombinációik.
\begin{verbatim*}
\put(0,55){\framebox(40,15)[lt]{Bal felső}}
\put(45,55){\framebox(40,15)[t]{Közép felső}}
\put(90,55){\framebox(40,15)[rt]{Jobb felső}}
\put(0,35){\framebox(40,15)[l]{Bal közép}}
\put(45,35){\framebox(40,15){Közép}}
\put(90,35){\framebox(40,15)[r]{Jobb közép}}
\put(0,15){\framebox(40,15)[lb]{Bal közép}}
\put(45,15){\framebox(40,15)[b]{Közép}}
\put(90,15){\framebox(40,15)[rb]{Jobb közép}}
\end{verbatim*}
\vfill

\pagebreak % ---------------------------------------------------------------------------------------------------------------------

\setlength{\unitlength}{1mm}
\begin{picture}(100,75)
\thicklines
%\thinlines
\put(0,55){\framebox(40,15)[lt]{Bal felső}}
\put(45,55){\framebox(40,15)[t]{Közép felső}}
\put(90,55){\framebox(40,15)[rt]{Jobb felső}}
\put(0,35){\framebox(40,15)[l]{Bal közép}}
\put(45,35){\framebox(40,15){Közép}}
\put(90,35){\framebox(40,15)[r]{Jobb közép}}
\put(0,15){\framebox(40,15)[lb]{Bal közép}}
\put(45,15){\framebox(40,15)[b]{Közép}}
\put(90,15){\framebox(40,15)[rb]{Jobb közép}}
\end{picture}
\end{itemize}
\vfill
\end{frame}

% ----------------------------------------------------------------------------------------------------------------------------------------

\begin{frame}[fragile,squeeze,allowframebreaks]
\frametitle{Táblázatok}

\begin{itemize}
\item Táblázatok, több soros és oszlopos cellák.
\begin{verbatim*}
\begin{tabular}{c|c|c}
A & B & C \\
\hline
a & b & c \\
d & e & f \\    
\end{tabular}
\end{verbatim*}
%\begin{table}
%\centering
\begin{tabular}{c|c|c}
A & B & C \\
\hline
a & b & c \\
d & e & f \\    
\end{tabular}
%\caption{Magyarázó szöveg.}
%\label{tab:pelda}
%\end{table}
\item Több soros cellák: 
\begin{verbatim*}
\multicolumn{2}{c|}{szöveg}
\end{verbatim*}
\end{itemize}
\vfill

\pagebreak % ---------------------------------------------------------------------------------------------------------------------

\begin{itemize}
\item Több soros és oszlopos cellák: \textup{multirow} környezet.
\item Hosszú táblázatok: \textit{longtable}
\item Másképpen működik a \textit{multirow} és a \textit{multicolumn} környezet.
\end{itemize}
\vfill
\end{frame}

% ----------------------------------------------------------------------------------------------------------------------------------------

\begin{frame}
%\frametitle{Minden jó, ha a vége jó}

\vfill
\begin{center}
\textbf{\huge Vége}

\bigskip
Köszönöm a figyelmet!
\end{center}
\vfill
\end{frame}

\end{document}
