\documentclass[t,aspectratio=169]{beamer}

\usepackage{soul}
\usepackage[utf8]{inputenc}

% ----------------------------------------------------------------------------------------------------------------------------------------

\title{\hfill \includegraphics{MFE.png}\\Bevezetés a \LaTeX\ használatába}

% \subtitle{1-4. előadás}

\author{Dr.~Facskó Gábor, PhD\\\tiny főiskolai adjunktus, tudományos főmunkatárs\\\textit{facsko.gabor@uni-milton.hu}}

\institute{\Tiny Milton Friedman Egyetem, Informatikai Tanszék, 1039 Budapest, Kelta utca 2.\\Wigner Fizikai Kutatóközpont, Űrfizikai és Űrtechnikai Osztály, 1121 Budapest, Konkoly-Thege Miklós út 29-33.\\ \texttt{https://wigner.hu/$\sim$facsko.gabor}}

\date{2024.~február~24.}

\logo{\includegraphics[height=0.9 true cm]{kekcsik.png}}

% ----------------------------------------------------------------------------------------------------------------------------------------

\begin{document}

\frame{\titlepage}

% ----------------------------------------------------------------------------------------------------------------------------------------

\begin{frame}
\frametitle{A tárgy célja}
\begin{itemize}
\setlength{\parskip}{0pt}
\setlength{\itemsep}{0pt}
\item Megismertetni a hallgatókkal egy platformfüggetlen, gyors, szabad tördelőszerkesztő programot
\item Megmutatni, hogy van élet az MS Office-n és az OpenOffice-n túl
\item Megkönnyíteni a laborjegyzőkönyv, szakdolgozat és diplomamunka írást
\end{itemize}
\end{frame}

% ----------------------------------------------------------------------------------------------------------------------------------------

\begin{frame}
\frametitle{A tárgy teljesítésnek követelményei}
\begin{enumerate}
\setlength{\parskip}{0pt}
\setlength{\itemsep}{0pt}
\item Laborgyakorlat, de nem kötelező bejárni levezősöknek. 
\item Kapcsolódjanak be az online előadásba, ta\-nul\-má\-nyoz\-zák a slide-okat, konzultáljanak, olvassák el a köny\-vet, hallgassák meg a felvételeket
\item Tanulmányozzák az interneten található oldalakat
 \item Küldjenek be a félév végéig két beadandó feladatot
\begin{enumerate}
\item A szakdolgozatukat, vagy egy korábbi önálló labor jegyzőkönyvüket, esetleg egy rövidebb szöveget írják át \LaTeX-be és küldjék el az előadónak
\item Az előző feladat nehézségével fordítottan a\-rá\-nyo\-san kijelöl néhány oldalt és azt be kell gépelni és beküldeni
\end{enumerate}
\end{enumerate}
Ajánlott olvasmány: \textit{Wettl Ferenc-Mayer Gyula-Sudár Csaba: \LaTeX\ kezdőknek és haladóknak} 
\end{frame}

% ----------------------------------------------------------------------------------------------------------------------------------------

\begin{frame}
\frametitle{Alapfogalmak, bevezetés}
\begin{itemize}
\setlength{\parskip}{0pt}
\item Szövegszerkesztő: notepad, vi, emacs, joe
\item Fejlettebb: MS Office, OpenOffice
\item Tördelőszerkesztő: képes átfogóan formázni a szö\-ve\-get, ábrákat, táblázatokat pozicionálni, tipográfiai sza\-bá\-lyo\-kat vesz figyelembe, ismeri a lingatúrákat
\item \TeX\ tördelőszerkesztő. Nyomdakész anyagokat produkál. Könnyű matematikai képleteket és hosszabb szöveget, könyvet és cikket készíteni vele (Kifejlesztése: \textit{Donald Ervin Knuth: A számítógép-prog\-ra\-mo\-zás művészete}.)
\end{itemize}
\end{frame}

% ----------------------------------------------------------------------------------------------------------------------------------------

\begin{frame}
\frametitle{Alapfogalmak, bevezetés - cont'd}
\begin{itemize}
\setlength{\parskip}{0pt}
\item \LaTeX\ egy \TeX\ makró csomag, amely lényegesen megkönnyíti a programrendszer használatát
\item \st{dvi file: platform independent file. Megn{\'e}zhet{\H{o}} (xdvi), nyomtahat{\'o} form{\'a}tumm{\'a} alak{\'\i}that{\'o}, vagy postscript{\'e}, encapsulated postscript{\'e} konvert{\'a}lhat{\'o}}\\Manapság már pdflatex-hel Portable Document Formatot (pdf) hoz létre a \LaTeX\ fordító
\item A postscript egy speciális nyomtató nyelv
\item Az encapsulated postscript a nyomtató nyelv keretes, egy oldalas változata
\end{itemize}
\end{frame}

% ----------------------------------------------------------------------------------------------------------------------------------------

\begin{frame}
\frametitle{Alapfogalmak, bevezetés - cont'd}
\begin{itemize}
\setlength{\parskip}{0pt}
\setlength{\itemsep}{0pt}
\item A \LaTeX\ előnyei 
\begin{itemize}
\item Nagyon szép, nyomdakész szöveget állít elő
\item Gyors, ingyenes, platformfüggetlen
\item Nem kell az ábrák, táblázatok elhelyezésével tö\-rőd\-ni
\item Könnyebb képletet szerkeszteni benne
\item Lehet benne levelet, prezentációt, posztert, cikket, disszertációt készíteni (de nem muszáj, sőt!)
\end{itemize}
\item A \LaTeX\ hátrányai
\begin{itemize}
\item Néhány oldalas szöveg és prezentáció készítés esetén lassabb és kö\-rül\-mé\-nye\-sebb, mint a meg\-szo\-kott szö\-veg\-szer\-kesz\-tők
\end{itemize}
\end{itemize}
%\vfill
\end{frame}

% ----------------------------------------------------------------------------------------------------------------------------------------

\begin{frame}
\frametitle{Alapfogalmak, bevezetés - cont'd}
\begin{itemize}
\setlength{\parskip}{0pt}
\setlength{\itemsep}{0pt}
\item Szerkesztési ciklus: 
\bigskip
\begin{list}{}{}
\item \framebox{Forrás file}
\item $\downarrow$
\item \framebox{Fordítás}
\item $\downarrow$ 
\item \framebox{dvi file}  $\rightarrow$ \framebox{Nyomtatás}
\item $\downarrow$
\item \framebox{Megjelenítés} 
\end{list}
\bigskip
Vagy amit akarunk\dots
\end{itemize}
%\vfill
\end{frame}

% ----------------------------------------------------------------------------------------------------------------------------------------

\begin{frame}
\frametitle{Alapfogalmak, bevezetés - cont'd}
\begin{itemize}
\setlength{\parskip}{0pt}
\item MS Windows: MiKTeX keret program
\item UNIX/Linux: tetex programcsalád 
\begin{list}{+}{}
\item emacs szövegszerkesztő
\item auctex (emacs) makrógyűjtemény
\item vi/vim szövegszerkesztő
\item Utófeldolgozás:
\begin{list}{-}{}
\item dvi2ps: postscriptre konvertál
\item dvi2pdf: Pdf file konvertál
\item pdfunify: egyesíti a pfileokat
\end{list}
\end{list}
\end{itemize}
%\vfill
\end{frame}

% ----------------------------------------------------------------------------------------------------------------------------------------

\begin{frame}
\frametitle{Első szöveg megszerkesztése, fordítása, megjelenítése}
\begin{list}{\texttt{facsko@elendil$\sim$\$}}{}
\setlength{\parskip}{0pt}
\setlength{\itemsep}{0pt}
\item emacs elso.tex
\end{list}

\textbackslash documentclass[a4paper,12pt]\{article\}

\textbackslash begin \{document \}\\
Megy ez.\\
\textbackslash end\{document\}

\begin{list}{\texttt{facsko@elendil$\sim$\$}}{}
\setlength{\parskip}{0pt}
\setlength{\itemsep}{0pt}
\item latex elso.tex
\item xdvi elso.dvi
\item dvips -o elso.ps elso.dvi
\item dvipdf elso.dvi elso.pdf
\item xpdf elso.pdf
\item acroread elso.pdf
\\ 
\item pdflatex elso.tex
\item xpdf elso.pdf
\end{list}
%\vfill
\end{frame}

% ----------------------------------------------------------------------------------------------------------------------------------------

\begin{frame}
\frametitle{Betűtípusok}

\textbackslash documentclass[a4paper,12pt]\{article\}

\textbackslash begin \{document \}\\
Megy ez.\\
\textbackslash textbf\{Megy ez.\}\\
\textbackslash textit\{Megy ez.\}\\
\textbackslash texttt\{Megy ez.\}\\
\textbackslash textsc\{Megy ez.\}\\
\textbackslash end\{document\}
%\vfill
\end{frame}

% ----------------------------------------------------------------------------------------------------------------------------------------

\begin{frame}
\frametitle{Speciális karakterek}

Próbáljuk ki: \%, \{, \}, $\sim$, \_

Hibaüzenetek.
\vfill
\end{frame}

% ----------------------------------------------------------------------------------------------------------------------------------------

\begin{frame}
\frametitle{Magyarítás}

\textbackslash documentclass[a4paper,12pt]\{article\}

\textbackslash begin \{document \}\\
Az én második szövegem.\\
\textbackslash end\{document\}

\textrm{Az n msodik szvegem.}

\texttt{\textbackslash usepackage[utf8]\{inputenc\}}
%\vfill
\end{frame}

% ----------------------------------------------------------------------------------------------------------------------------------------

\begin{frame}
\frametitle{Bekezdések, sorok, listák}

\begin{itemize}
\item Bekezdések formázása
\setlength{\parskip}{2pt}
\begin{list}{}{}
\setlength{\itemsep}{0pt}
\item \textbackslash parindent: Behúzás a bekezdés elején
\item \textbackslash noindent: Behúzás kikapcsolása
\item \textbackslash parskip: Kihagyás a bekezdések között
\end{list}

\item Sortörés, oldaltörés
\setlength{\parskip}{2pt}
\begin{list}{}{}
\setlength{\itemsep}{0pt}
\item \textbackslash\textbackslash: sortörés
\item \textbackslash pagebreak: oldaltörés
\end{list}

\item Listák
\setlength{\parskip}{2pt}
\begin{list}{}{}
\setlength{\itemsep}{0pt}
\item \textbackslash begin\{itemize\}
\item \textbackslash begin\{enumerate\}
\item \textbackslash begin\{list\}
\end{list}
\end{itemize}
%\vfill
\end{frame}

% ----------------------------------------------------------------------------------------------------------------------------------------

\begin{frame}
\frametitle{Képek és elhelyezésük}
\begin{itemize}  
\item Csomag és utasítás:
\setlength{\parskip}{-5pt}
\begin{list}{}{}
\setlength{\itemsep}{0pt}
\item \textbackslash usepackage[final]\{graphicx\}
\item \textbackslash includegraphics[width=300pt]\{jgr-2014-year-f03.eps\}
\end{list}
\item Használjuk a \textit{figure} környezetet:
\setlength{\parskip}{-2pt}
\begin{list}{}{}
\setlength{\itemsep}{0pt}
\item \textbackslash begin\{figure\}[h]
\item \textbackslash centering
\item \textbackslash caption\{Magyarázó szöveg\}
\item \textbackslash label\{fig:címke\}
\item \textbackslash end\{figure\}
\end{list}
\item A \textit{figure$*$} környezet elhagyja az automatikus sor\-szá\-mo\-zást, vagy teljes szé\-les\-ség\-ben helyezi el az ábrát

\item Az ábrák és a táblázatok ún.~lebegő objektumok és a \LaTeX\ irányítja, hogy hova kerüljenek. Minimális irányítás:  h - helyben, t - top, b - bottom, ! - nyomtékosítás
\end{itemize}
%\vfill
\end{frame}

% ----------------------------------------------------------------------------------------------------------------------------------------

\begin{frame}
\frametitle{Táblázatok, tabulátorok}
\setlength{\parskip}{6pt}
\begin{list}{}{}
\setlength{\itemsep}{0pt}
\item \textbackslash begin\{table\}[h]
\item \textbackslash centering
\item \textbackslash begin\{tabular\}[c]{clh}
\item Első \& Második \& Harmadik oszlop \textbackslash\textbackslash
\item \textbackslash end\{tabular\}
\item \textbackslash caption\{Magyarázó szöveg\}
\item \textbackslash label\{tab:tablazat\}
\item \textbackslash end\{table\}
\end{list}
Több oszlop összevonása: \textit{multicolumn} parancs\\
Több sor összevonása: \textit{multirow} parancs\\
Hosszú táblázatok: \textit{longtable} környezet\\
%\vfill
\end{frame}

% ----------------------------------------------------------------------------------------------------------------------------------------

\begin{frame}
\frametitle{Címkék, hivatkozások, számlálók}
\setlength{\parskip}{6pt}
\begin{itemize}
\setlength{\itemsep}{0pt}
\item A címke (label) egy viszonyítási pont a szövegben:\\ \texttt{\textbackslash label\{fig:abra\}}
\item A hivatkozás a címkére hivatkozik:\\ \texttt{\textbackslash ref\{fig:abra\}}
\item Képre, bekezdésre, szövegre, táblázatra ugyanúgy hivatkozunk. Az előtag csak konvenció, nem kötelező
\item Mind a listák számát, a fejezeteket, a lábjegyzeteket, az ábrák és táblázatok számát a \LaTeX\ nyilvántartja. A változók neve: számláló.
\item A számlálók értékét módosítani lehet:\\ \textbackslash setcounter\{enumi\}\{4\}
\end{itemize}
%\vfill
\end{frame}

% ----------------------------------------------------------------------------------------------------------------------------------------

\begin{frame}
\frametitle{Dokumentum osztályok}
\setlength{\parskip}{6pt}
\begin{itemize}
\setlength{\itemsep}{0pt}
\item Book, report, slides, article
\item \textbackslash section\{cím\}, \textbackslash section$*$\{cím\}
\item \textbackslash subsection\{cím\},\textbackslash subsection$*$\{cím\} 
\item \textbackslash subsubsection\{cím\}, \textbackslash subsubsection$*$\{cím\}
\item \textbackslash paragraph\{cím\}
\item \textbackslash subparagraph\{cím\}
\item \textbackslash chapter\{cím\}, \textbackslash chapter$*$\{cím\}
\end{itemize}

Feladat: kipróbálni az eddigi szövegek viselkedését, a fenti dokumentum osztályokkal. Milyen paraméterei vannak a különféle osztályoknak?
%\vfill
\end{frame}

% ----------------------------------------------------------------------------------------------------------------------------------------

\begin{frame}
%\frametitle{Minden jó, ha a vége jó}

\vfill
\begin{center}
\textbf{\huge Vége}

\bigskip
Köszönöm a figyelmet!
\end{center}
\vfill
\end{frame}

\end{document}
