\documentclass[aspectratio=169]{beamer}

\usepackage{cite}
\usepackage{esint}
%\usepackage{natbib}
\usepackage{hologo}
\usepackage{inlinebib}

% ----------------------------------------------------------------------------------------------------------------------------------------

\title{\hfill \includegraphics{MFE.png}\\Bevezetés a \LaTeX\ használatába}

% \subtitle{5-8. előadás}

\author{Dr.~Facskó Gábor, PhD\\\tiny főiskolai adjunktus, tudományos főmunkatárs\\\textit{facsko.gabor@uni-milton.hu}}

\institute{\Tiny Milton Friedman Egyetem, Informatikai Tanszék, 1039 Budapest, Kelta utca 2.\\Wigner Fizikai Kutatóközpont, Űrfizikai és Űrtechnikai Osztály, 1121 Budapest, Konkoly-Thege Miklós út 29-33.\\ \texttt{https://wigner.hu/$\sim$facsko.gabor}}

\date{2024.~április~6.}

\logo{\includegraphics[height=0.9cm]{kekcsik.png}}

% ----------------------------------------------------------------------------------------------------------------------------------------

\begin{document}

\frame{\titlepage}

% ----------------------------------------------------------------------------------------------------------------------------------------

\begin{frame}[fragile,squeeze,allowframebreaks]
\frametitle{Matematikai szövegek szerkesztése matematikai módban}
\begin{itemize}
\item Szövegközi képletek: \$ \texttt{F\_i} \ \$  \hskip 2 cm $F_i$
\item Képlet külön sorban: \$\$ \texttt{F\_i} \$\$  \[ F_i \]
\item Szabványos \LaTeX:\\
\textbackslash( \texttt{F\_i} \textbackslash) \( F_i \)  \\
\textbackslash[ \texttt{F\_i} \textbackslash] \hskip 2 cm \[F_i\]
\item A \textit{math} környezet szövegközi képleteket kezel:
\tiny
\begin{verbatim*}
A cserebogarak tapogatóinak hossza \\ 
\begin{math} \\
\frac{\sqrt{a^3}}{5} \\
\end{math}
mérhetö.
\end{verbatim*}
\normalsize
A cserebogarak tapogatóinak hossza
\begin{math}
\frac{\sqrt{a^3}}{5}
\end{math}
mérhető.
\end{itemize}
\vfill

\pagebreak % --------------------------------------------------------------------------------------------------------------------

\begin{itemize}
\setlength{\itemsep}{0pt}
\item Az \textit{equation} környezet:
%\small
\begin{verbatim*}
\begin{equation} 
\frac{\sqrt{a^3}{5}}
\end{equation*}
\end{verbatim*}
\begin{equation}
\frac{\sqrt{a^3}}{5}
\end{equation}

\pagebreak % --------------------------------------------------------------------------------------------------------------------

\item Az \textit{eqnarray} környezet:
\begin{verbatim*}
\begin{eqnarray}
c & = & \frac\{\sqrt\{a^3}}{b} \\
F_L & = & q \cdot (\mathbf{v} \times \mathbf{B}) \\
E &=& m c^2
\end{eqnarray}
\end{verbatim*} 
%\normalsize
\begin{eqnarray}
\setcounter{equation}{1}
c &=& \frac{\sqrt{a^3}}{b} \\
F_L &=& q \cdot \left(\mathbf{v} \times \mathbf{B}\right)\\
E &=& m c^2
\end{eqnarray}

\pagebreak % --------------------------------------------------------------------------------------------------------------------

\item A \textit{eqnarray*} környezet:
\begin{verbatim*}
\begin{eqnarray*}
c &=& \frac{\sqrt{a^3}}{b} \\
F_L &=& q \cdot\left(\mathbf{v} \times \mathbf{B}\right) \\
E &=& m c^2 \\
\end{eqnarray*}
\end{verbatim*} 
\begin{eqnarray*}
\setcounter{equation}{1}
c &=& \frac{\sqrt{a^3}}{b}\\
F_L &=& q \cdot \left(\mathbf{v} \times \mathbf{B}\right)\\
E &=& m c^2
\end{eqnarray*}
\end{itemize}
\vfill

\pagebreak % --------------------------------------------------------------------------------------------------------------------

\begin{itemize}
\item Négyzetgyökvonás: \textbackslash sqrt\{11\}  $\rightarrow\sqrt{11}$
\item Gyökvonás: \textbackslash sqrt[3]\{b\^{}4\} $\rightarrow\sqrt[3]{b^4}$
\item Törtek: \textbackslash frac\{a\}\{b\} $\rightarrow\frac{a}{b}$
\item Sinus és cosinus: \textbackslash sin és \textbackslash cos
\item Szimbólumok: 
\begin{itemize}
\item +, -, $<$, $>$, =
\item \textbackslash le, \textbackslash ge, \textbackslash ne $\rightarrow\le, \ge, \ne$
\item \textbackslash sim $\rightarrow \sim$
\item A és B halmaz metszete: A\textbackslash cap B $\rightarrow A\cap B$
\item A és B halmaz uniója: A\textbackslash cup B $\rightarrow A\cup B$
\item Görög betűk: \textbackslash alpha$\rightarrow\alpha$, \textbackslash beta$\rightarrow\beta$, \textbackslash Gamma$\rightarrow\Gamma$, \textbackslash Delta $\rightarrow\Delta$, \textbackslash Theta, \textbackslash theta, \textbackslash vartheta, $\rightarrow\Theta, \theta, \vartheta$
\item Differenciál operátorok: \textbackslash nabla $\rightarrow\nabla$
\end{itemize}
\end{itemize}
\texttt{https://oeis.org/wiki/List\_of\_LaTeX\_mathematical\_symbols}
\vfill

\pagebreak % --------------------------------------------------------------------------------------------------------------------

\begin{itemize}
\item Parciális differenciálás:\\\textbackslash partial x\_i $\rightarrow\partial x_i$\\\textbackslash left(\textbackslash frac\{\textbackslash partial\}\{\textbackslash partial x\_i\}\textbackslash right) $\rightarrow\left(\frac{\partial}{\partial x_i}\right)$\\ \textbackslash left(\textbackslash frac\{\textbackslash partial\^{}2\}\{\textbackslash partial\^{}2 x\_i\}\textbackslash right)$\rightarrow\left(\frac{\partial^2}{\partial^2 x_i}\right)$\\
\item Integrálok kiszámítása
\begin{itemize}
\item Határozatlan integrál:\\\textbackslash int \textbackslash frac\{1\}\{x\} dx$\rightarrow\int \frac{1}{x} dx$
\item Határozott integrál:\\\textbackslash int\textbackslash limits\_\{-\textbackslash infty\}\^{}\{\textbackslash infty\} \textbackslash frac\{\textbackslash sin x\}\{x\} dx=1 $\rightarrow\int\limits_{-\infty}^{\infty} \frac{\sin x}{x} dx=1$
\item Többszörös integrál (\textit{esint} csomagban):\\\textbackslash iiint g(x,y,z) \textbackslash,dx\textbackslash,dy\textbackslash,dz $\rightarrow\iiint g\left(x,y,z\right) \,dx\,dy\,dz$
\end{itemize}
\end{itemize}
\vfill

\pagebreak % --------------------------------------------------------------------------------------------------------------------

\begin{itemize}
\item Integrálok kiszámítása - cont'd
\begin{itemize}
\item Körintegrál(\textit{esint} csomagban):\\ \textbackslash oiint\textbackslash limits\_\{S\} F\textbackslash left(x,y\textbackslash right) \textbackslash,dx\textbackslash,dy\\$\rightarrow \oiint\limits_{S} F\left(x,y\right) \,dx\,dy$
\end{itemize}
\item Összegek kiszámítása:\\\textbackslash sum\^{}n\_\{i=1\} \textbackslash frac\{1\}\{x\_i\} $\rightarrow\sum_{n}^{i=1} \frac{1}{x_i}$\\\textbackslash sum\textbackslash limits\^{}n\_\{i=1\} \textbackslash frac\{1\}\{x\_i\} $\rightarrow\sum\limits^{n}_{i=1} \frac{1}{x_i}$
\item Szorzatok kiszámítása:\\\textbackslash prod\^{}n\_\{i=1\} x\_i\ $\rightarrow\prod_{n}^{i=1} x_i$\\\textbackslash prod\textbackslash limits\^{}n\_\{i=1\} x\_i\ $\rightarrow\prod\limits^{n}_{i=1} x_i$
\end{itemize}
\vfill

\pagebreak % --------------------------------------------------------------------------------------------------------------------

\begin{itemize}
\item Határértékek kiszámítása:\\\textbackslash lim\textbackslash limits\_\{n \textbackslash to \textbackslash infty\} a\_n $\rightarrow\lim\limits_{n \to \infty} a_n$

\item Unió halmazrendszer:\\\textbackslash bigcup\textbackslash limits\_\{i=1\}\^{}\{n\} H\_\{i\} $\rightarrow\bigcup\limits_{i=1}^{n} H_{i}$
\item Metszet halmazrendszer:\\\textbackslash bigcap\textbackslash limits\_\{i=1\}\^{}\{n\} H\_\{i\} $\rightarrow\bigcap\limits_{i=1}^{n} H_{i}$
\end{itemize}
\vfill
\end{frame}

% ----------------------------------------------------------------------------------------------------------------------------------------

\begin{frame}[fragile,squeeze]
\frametitle{Gyakorló feladatok}

\begin{enumerate}
\item Szerkesszék meg az alábbi képleteket:
\begin{eqnarray*}
\sin 2 x &=& 2 \sin x \cos x\\
\cos 2 x &=& \cos^2 x - \sin^2 x\\
\frac{\partial\rho}{\partial t}+\nabla\left(\rho\mathbf{v}\right)&=&0
\end{eqnarray*}
\item Másodfokú egyelet megoldóképletének levezetése.
\item Írja fel a Maxwell egyenletek integrális alakját!
\item Írja fel a Stokes-tételt!
\item Írja fel a Gauss-Ostrogradskij tételt!
\end{enumerate}
\vfill
\end{frame}

% ----------------------------------------------------------------------------------------------------------------------------------------

\begin{frame}[fragile,squeeze]
\frametitle{Irodalomjegyzék létrehozása a szövegben}
Az önsanyargató hajlamokkal rendelkezők használják a \texttt{\textbackslash bibitem} parancsot a \texttt{thebibliography} környezetben:
\small
\begin{verbatim*}
\begin{thebibliography}{\textbf{Irodalomjegyzék}
\bibitem[Bacsardi(2005)]{bacsi2}Bacsardi, L.\ 2005, Acta Astronautica, 57, 224
}
\bibitem[Bacsardi(2007)]{bacsi1}Bacsardi, L.\ 2007, Acta Astronautica, 61, 151
\end{thebibliography}
\end{verbatim*}

Idézem Bacsárdi László összes művét: \textbackslash cite\{bacsi1,bacsi2\}.

%\medskip
%\textbf{Irodalomjegyzék}
%\begin{thebibliography}
%\item[Bacsardi(2005)]{bacsi1} Bacsardi, L.\ 2005, Acta Astronautica, 57, 224
%\item[Bacsardi(2007)]{bacsi2} Bacsardi, L.\ 2007, Acta Astronautica, 61, 151 
%\end{thebibliography}
%\vfill
\end{frame}

% ----------------------------------------------------------------------------------------------------------------------------------------

\begin{frame}[fragile,squeeze]
\frametitle{Irodalomjegyzék létrehozása a \hologo{BibTeX} segédprogrammal}

\begin{itemize}
\item Használata:
\begin{verbatim*}
\bibliography{facsko_gabor_researchplan} 
\bibliographystyle{plain}
\end{verbatim*}
\item Irodalomjegyzék formátumok: \texttt{plain}, \texttt{unsrt}, \texttt{alpha}, \texttt{abbvr}. Idézés: \texttt{\textbackslash cite\{\}}
\item A \textit{natbib} csomaggal pl.~\textit{plainnat} (ld.~a csomag paramétereit). Idézés: \texttt{\textbackslash citep\{\}} \texttt{\textbackslash citet\{\}} (paraméterek).
\item A \hologo{BibTeX} file tartalma: \texttt{article}, \texttt{book}, \texttt{report}.
\item SAO/NASA Astrophysics Data System (ADS) a kész \hologo{BibTeX} formátumú bibliográfiai adatok lelőhelye:\\\texttt{http://adsabs.harvard.edu/}
\end{itemize}
\vfill
\end{frame}

% ----------------------------------------------------------------------------------------------------------------------------------------

\begin{frame}[fragile,squeeze,allowframebreaks]
\frametitle{Példák \hologo{BibTeX} formátumokra}

\begin{verbatim*}
@ARTICLE{facsko08:_clust,
   author = {{Facsk{\'o}}, G. and {Kecskem{\'e}ty}, K. and 
{Erd{\H o}s}, G. and {T{\'a}trallyay}, M. and 
{Daly}, P.~W. and {Dandouras}, I.},
    title = "{A statistical study of hot flow 
anomalies using Cluster data}",
  journal = {Advances in Space Research},
     year = 2008,
   volume = 41,
    pages = {1286-1291},
      doi = {10.1016/j.asr.2008.02.005}
}
\end{verbatim*}
\vfill

%\pagebreak % --------------------------------------------------------------------------------------------------------------------

\begin{verbatim*}
@Book{laakso10:_clust_activ_archiv,
  editor = 	 {{Laakso}, H. and {Taylor}, M. 
and {Escoubet}, C.~P.}
  title = 	 {The Cluster Active Archive},
  publisher = 	 {Springer Science+Business Media B.V.},
  year = 	 2010,
  note = {Astrophysics and Space Science Proceedings, 
ISBN 978-90-481-3498-4}},
  doi = {10.1007/978-90-481-3499-1}
}
\end{verbatim*}
\vfill

%\pagebreak % --------------------------------------------------------------------------------------------------------------------

\begin{verbatim*}
@InProceedings{facsko10:_clust_hot_flow_anomal_obser,
   author = {{Facsk{\'o}}, G. and {T{\'a}trallyay}, M. and 
{Erd{\H o}s}, G. and {Dandouras}, I.},
    title = "{Cluster Hot Flow Anomaly Observations 
During Solar Cycle Minimum}",
booktitle = {The Cluster Active Archive, Studying 
the Earth's Space Plasma Environment},
     year = 2010,
   editor = "{Laakso, H., Taylor, M., \& Escoubet, C.~P.}",
    pages = {369-375},
      doi = {10.1007/978-90-481-3499-1\_25}
}
\end{verbatim*}
\vfill
\end{frame}

% ----------------------------------------------------------------------------------------------------------------------------------------

\begin{frame}
%\frametitle{Minden jó, ha a vége jó}

\vfill
\begin{center}
\textbf{\huge Vége}

\bigskip
Köszönöm a figyelmet!
\end{center}
\vfill
\end{frame}

\end{document}
